\section*{References}

\begin{thebibliography}{99}
\bibitem{1} L. Ahlfors, \textit{The theory of meromorphic curves}, Acta Soc. Sci. Fenn. Nova Ser. A 3 (1941), no. 4, 1-31.
\bibitem{2} M. F. Atiyah and I. M. Singer, \textit{The Index of elliptic operators: III}, Ann. of Math. 87 (1968), 546-604.
\bibitem{3} A. Avez, \textit{Characteristic classes and Weyl tensor: applications to general relativity}, Proc. Nat. Acad. Sci. (USA) 66 (1970), 265-268.
\bibitem{4} P. F. Baum, \textit{Vector fields and Gauss-Bonnet}, Bull. Amer. Math. Soc. 76 (1970), 1202-1211.
\bibitem{5} P. F. Baum and R. Bott, \textit{On the zeroes of meromorphic vector fields, Essays on topology and related topics dedicated to G. de Rham}, 1970, 29-47.
\bibitem{6} P. F. Baum and J. Cheeger, \textit{Infinitesimal isometries and Pontrjagin numbers}, Topology 8 (1969), 173-193.
\bibitem{7} A. Borel and F. Hirzebruch, \textit{Characteristic classes and homogeneous spaces}, Amer. J. Math. 80 (1958), 458-538; 81 (1959), 315-382; 82 (1960), 491-504.
\bibitem{8} R. Bott, \textit{Vector fields and characteristic numbers}, Michigan Math. J. 14 (1967), 231-244.
\bibitem{9} , \textit{A residue formula for holomorphic vector fields}, J. Diff. Geom. 1 (1967), 311-330.
\bibitem{10} R. Bott, \textit{On a topological obstruction to integrability}, Proc. Symp. in Pure Math., Amer. Math. Soc. 16 (1970), 127-131.
\bibitem{11} H Cartan, \textit{Notion d'algebre differentielle; applications aux groupes de Lie et aux variétés ou opère un groupe de Lie}, Coll. de Topologie, Bruxelles (1950), 15-27.
\bibitem{12} , \textit{La transgression dans un groupe de Lie et dans un espace fibré principal}, ibid, 57-71.
\bibitem{13} S. Chern, \textit{Characteristic classes of hermitian manifolds}, Annals of Math. 47 (1946), 85-121.
\bibitem{14} , \textit{Topics in differential geometry}, Inst. for Adv. Study, Princeton, 1951.
\bibitem{15} S. Chern, \textit{Differential geometry of fiber bundles}. Int. Cong. of Math., II (1950), 397-411.
\bibitem{16} ______, \textit{Complex manifolds without potential theory}, van Nostrand 1967; also, this book.
\bibitem{17} ______, \textit{Holomorphic curves in the plane}, Differential geometry In honor of Yano, Tokyo, 1972, 73-94.
\bibitem{18} ______, and J. Simons, \textit{Some cohomology classes in principal fiber bundles and their application to riemannian geometry}, Proc. Nat. Acad. Sci. (USA) 68 (1971), 791-794.
\bibitem{19} ______, and J. Simons, \textit{Characteristic forms and geometrical invariants}, Annals of Math 99 (1974), 48-69.
\bibitem{20} L. P. Eisenhart, \textit{Riemannian geometry}, Princeton, 1949.
\bibitem{21} F. Hirzebruch, \textit{Topological methods in algebraic geometry}, Springer, 1966.
\bibitem{22} H. Hopf, \textit{Vektorfelder in Manufgaltigkeiten}, Math. Annalen 96 (1927), 225-250.
\bibitem{23} D. Husemoller, \textit{Fibre bundles}, McGraw-Hill, 1966.
\bibitem{24} S. Kobayashi and K. Nomizu, \textit{Foundations of differential geometry}, vol. 2, Interscience, 1969.
\bibitem{25} S. Kobayashi and T. Ochiai, \textit{G-structures of order two and transgression operators}, J. Diff. Geom. 6 (1971), 213-230.
\bibitem{26} J. Milnor and J. D. Stasheff, \textit{Characteristic classes}, Annals of Math. Studies 76, Princeton, 1974.
\bibitem{27} L. Pontrjagin, \textit{Some topological invariants of closed riemannian manifolds}, Izvestiya Akad. Nauk SSSR Ser. Mat. 13 (1949), 125-162.
\bibitem{28} N. Steenrod, \textit{The topology of fibre bundles}, Princeton 1951.
\bibitem{29} E. Stiefel, \textit{Richtungsfelder und Fernparallelismus in Manufg-faltigkeiten}, Comm. Math. Helv. 8 (1936), 3-51.
\bibitem{30} W. Stoll, \textit{Value distribution of holomorphic maps into compact complex manifolds}, Springer notes vol. 135 (1970).
\bibitem{31} H. Whitney, \textit{On the topology of differentiable manifolds}, Lectures in Topology, Univ. of Michigan Press, 1941.
\bibitem{32} Hung-Hsi Wu, \textit{The equidistribution theory of holomorphic curves}, Annals of Math. Studies 64, Princeton, 1970.
\bibitem{33} Wen-Tsun Wu, \textit{Sur les espaces fibrés}, Act. Sci. et Indus. 1183 (1952).
\bibitem{34} S. Chern, \textit{Meromorphic vector fields and characteristic numbers}, Scripta Math. 29 (1973), 243-251.
\bibitem{35} R. Bott and A. Haefliger, \textit{On characteristic classes of $\Gamma$-foliations}, Bull. AMS 78 (1972), 1039-1044.
\bibitem{36} J. Carlson and P. Griffiths, \textit{A defect relation for equidimensional holomorphic mappings between algebraic varieties}, Annals of Math. 95 (1972), 557-584.
\bibitem{37} M. Cowen and P. Griffiths, \textit{Holomorphic curves and metrics of negative curvature}, J d'Analyse Math 29 (1976), 93-153.
\end{thebibliography}

\section*{Index}

\begin{itemize}
\item Abelian variety 3
\item Admissible (connection) $ \flat \natural $
\item Almost complex 12
\item Bianchi Identity 37,111
\item Bott's Theorem (on zeroes of holomorphic vector fields) 135
\item Calabi–Eckmann manifold $ \flat $
\item Canonical bundle 54
\item Cayley – Plücker – Grassmann coordinates 70
\item Characteristic class 99
\item Characteristic function (of Nevanlinna) 88
\item Chart 31
\item Chern class 34,76
\item Classifying space 74,76
\item Compensating function 88
\item Complex Manifold 1
\item Complex Projective Space 1
\item Complex structure 6
\item Complex torus 3
\item Complex vector bundle 31
\item Connection
- of vector bundle 35
- of fibre bundle 106
\item Connection matrix 36
\item Conjugate (complex structure) 9
\item Curvature form $ \flat \natural $
\item Curvature matrix 37,104
\item Covariant differential 38,102
\item Crofton type formula 146
\item Defect 146
\item Divisor $ \flat \natural $
\item Dolbeault groups 19
\item Dolbeault-Grothendieck lemma 20
\item Euler class 104
\item Euler – Poincaré formula 97
\item Exact sequence 25
\item Exhaustion function 86,144
\item Fiber 32
\item Fine (sheaf) 27
\item Flat (connection) 37
\item Foliation 118
\item Frame field 35
\item Gauss – Bonnet (theorem) 142
\item Grassmann manifold 69
\item Hermitian (complex manifold) 55
\item Hermitian – einsteinian (metric) 62
\item Hermitian structure 43
\item Hermitian vector bundle 43
\item Holomorphic vector bundle 46
\item Holomorphic curve 83
\item Holomorphic sectional curvature 61
\item Hopf fibering 2
\item Hopf manifold 3
\item Horizontal (sector) 36
\item Imbedding 63
\item Immersion 63
\item Integrable (almost complex structure) 16
\item Integrability condition 15
\item Restricted type (Kähleriam manifold) 63
\item Invariant (function) 38
\item Iwasawa manifold 4
\item Jacobi identity 108
\item Kähler form 10
\item Kählerian (hermitian manifold) 55
\item Killing equations 139
\item Lifting 36
\item Line bundle 32
\item Linearly equivalent (divisors) 48
\item Maurer - Cartan equations 79,106
\item Nevanlinna's defect relation 94
\item Order function 88,145
\item Partition of unity 27
\item Parallel vector field 36
\item Picard's theorem 94
\item Picard variety 54
\item Presheaf 24
\item Principal fibre bundle 108
\item Poincaré half plane 67
\item q-cochain 26
\item q-cocycle 26
\item Quotient sheaf 25
\item Resolution (fine) of a sheaf 30
\item Ricci form 62
\item Riemann conditions 66
\item Riemann surface 4
\item Schubert variety 70
\item Secondary invariants (of characteristic forms) 119
\item Sheaf 23
\item Sheaf homomorphism 25
\item Siegel upper half plane 66
\item Simons characters 131
\item Stiefel manifold 75
\item Stalk 23
\item Symmetric (invariant function) 39
\item Tensor product (of two vector bundles) 33
\item Torsion matrix 57
\item Transgression operator 116
\item Transition functions 32
\item Type (1,0) 46
\item Universal bundle 76
\item Universal line bundle 54
\item Vector field 36
\item Weil homomorphism 113
\item Whitney sum 33
\end{itemize}

\section*{Selected Papers}

\textbf{by Shling-shen Chern}

Chern's profound influence on the course of modern (differential) geometry is chronicled in this collection of papers that represent approximately one-third of his total work to date. Featuring many of his lesser known, previously inaccessible works, Chern's selections include papers on such topics as: projective differential geometry • euclidian geometry • geometrical structures and their intrinsic connections • integral geometry • characteristic classes • holomorphic mappings • minimal submanifolds • webs.  
In addition, there are introductory articles by André Weil, Philip Griffiths, and Chern's personal summary of his own mathematical work.

1978/xxxi, 476 pp./2 Portraits/Cloth  
ISBN 0-387-90339-9  

\section*{Elementary Topics in Differential Geometry}

\textbf{by J. Thorpe}

This new introductory text develops the geometry of $n$-dimensional oriented surfaces in $R^{n+1}$. By viewing such surfaces as level sets of smooth functions, the author is able to introduce global ideas early without the need for preliminary chapters developing sophisticated machinery. The calculus of vector fields is used as the primary tool in developing the theory. Coordinate patches are introduced only after preliminary discussions of geodesics, parallel transport, curvature, and convexity. Differential forms are introduced only as needed for use in integration. The text, which draws significantly on students' prior knowledge of linear algebra, multivariate calculus, and differential equations, is designed for a one-semester course at the junior/senior level.

1979/xiii, 256 pp./115 lllus./Cloth  
ISBN 0-387-90357-7

\section*{Several Complex Variables}

by H. Grauert and K. Fritzsche

An introduction to the modern theory of several complex variables with descriptions of the theory's most important definitions, concepts, and results. Central topics discussed include holomorphic functions in $C^n$ and on complex manifolds, regions of holomogy, power series algebras and the foundations of sheaf, and cohomology theory. A knowledge of calculus and the theory of functions of one variable is assumed.

1976/x, 182 pp./25 lllus./Cloth  
(Graduate Texts in Mathematics, Volume 38)  
ISBN 0-387-90172-8  

\section*{A Course in Differential Geometry}

by W. Klingenberg

A direct, classical oriented introduction to the theory of curves and planes as well as to differential geometry. Concepts and results of the global theory, the four-vertices-theorem, the Umlaufsatz, and the integral theorem of Gauss-Bonnet are the central subjects.

1978/xii, 192 pp./Cloth  
(Graduate Texts in Mathematics, Volume 51)  
ISBN 0-387-90255-4

\endinput